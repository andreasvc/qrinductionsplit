\documentclass{article}
\usepackage{fullpage}
\usepackage{amsmath}
\usepackage{verbatim}
\usepackage[pdftex]{graphicx}
\usepackage[english]{babel}
\usepackage[colorlinks=true, linkcolor=black, urlcolor=blue, pdfborder={0 0 0}]{hyperref}
%\usepackage{synttree}
%\usepackage[utf8x]{inputenc}

\title{Model Induction in Qualitative Reasoning}
\author{Andreas van Cranenburgh \\ 0440949 \\ acranenb@science.uva.nl 
\and Hanne Nijhuis \\ 0364568 \\ hnijhuis@science.uva.nl}

\begin{document}
\maketitle

\begin{abstract}
\end{abstract}

\section{Introduction}

\section{Literature Review}
In this section we will review a few of the papers that have contributed
greatly to the definition of Qualitative Reasoning. We will give a short
summary of the papers and will conclude with an overview of the current field
of research and possible future work.

\subsection{Qualitative Reasoning}
The following papers describe the foundations of qualitative reasoning.

\subsubsection{A qualitative Physics based on confluences (de Kleer \& Brown)}

\subsubsection{Qualitative Process Theory (Forbus)}

\subsubsection{Garp3 - Workbench for Qualitative Modelling and Simulation (Bredeweg et al.)}

\subsection{Automated Modeling}

To introduce the recent developments in the field of automated modelling we
will now review a few papers on this topic.

\subsubsection{Hylke}
Cluster, Superclusters, Causal paths.

\subsubsection{Jochem}
%less brute force, instead attempt to find the conceptually correct model immediately

\subsubsection{Carsten}
Causal groups % feedback loops, interacting processes

\subsection{Overview}
%overview of what?

\section{Theory}

In the field of QR, models %( / scenario's / systems ?) 
consist of
several smaller fragments which can be roughly divided in three categories:
static, process and agent. This categorization is not strict, but gives
a certain view on how things work. Static fragments are used to describe the
structure of a model, as well as proportionalities between quantities. A static
fragment cannot have any agents or influences in it. A \emph{process} fragment
should have at least one direct influence and is not restricted in terms of dependencies. %(relations?)
Finally, \emph{agent} fragments should be used to describe any influences on a
system that are external, hence not part of the system itself.

Fragments can be useful to make the complete system more comprehensible -- ie., they are a form of chunking. 
Also, while building these complex models, fragments allow the user to focus on
what happens in a small part of the model, without having to worry about the
other 'components'. For example in a model describing a population, seperate
fragments could describe processes like birth, death, emigration, etc.

Recent work on automated modeling \cite{buisman, vanweelden, liem} has focused
on generating a correct model based on a full-envisionment behavior graph. The
current algorithms always produce a single model fragment which describes the
complete system. The useful fragmentation of a system is hereby lost and
comprehending the output could become quite difficult with large models. We
therefore introduce a way of splitting such a single monolithic model into
smaller fragments while preserving the exact same behavior.

\section{Approach}
In this section we describe our approach on inducing model fragments.

\subsection{Monolithic model}

We take a monolithic model and divide it into several smaller fragments. To do
so, we need to comply to several requirements:

\begin{itemize}
\item all dependencies have to be retained

\item the behavior has to be reproducable

\item 

\end{itemize}

\subsection{Algorithm}

\begin{itemize}

\item Find a list of pivots, conditions on which the model fragments are based.
Currently these are single structural relations.

\item For each pivot, find a set of dependencies which apply whenever the pivot
is present, without exception

\item Generalize these dependencies into a single model fragment, several
dependencies among instances may be compressed into one

\end{itemize}

\section{Experiments \& Results}
\subsection{Evaluated Models}

% paste some output here, take screenshots or save EPS files from Garp3

\subsubsection{Tree and shade growth} 
% maybe show scenario with three trees, number of model fragments should stay the same! proof of concept.

\subsubsection{Stacked bath tubs}
\subsubsection{Communicating vessels} 
%both two and three? maybe also the truly compositional one?

\subsubsection{Population dynamics}
%single population? split into birth, migration etc.
%interacting populations? but ants garden is too much

\section{Future work}

% World domination etc.

\section{Conclusion}



\begin{thebibliography}{99}

% 1,5 - 2 pp. per artikel

%allebei
\bibitem{bredeweg-eco} Bredeweg, B. and Salles, P. (2009), Handbook of Ecological Modelling, Chapter 19 - Mediating conceptual knowledge using qualitative reasoning. In: J\/orgen, S.V., Chon, T-S., Recknagel, F.A. (Eds.), Handbook of Ecological Modelling and Informatics. Wit Press, Southampton, UK, pp. 351.398.

%allebei
\bibitem{bredeweg-garp} Bredeweg, B., Linnebank, F., Bouwer, A. and Liem, J. (2009)
02 Bredeweg EtAl ECOINF 2009.pdf (598.325 Kb)
Garp3 - Workbench for Qualitative Modelling and Simulation. Ecological Informatics 4(5-6), 263-281.

%hanne
\bibitem{forbus} Forbus, K.D. (1984)
Forbus1984.pdf (4.108 Mb)
Qualitative process theory. Artificial Intelligence, 24:85-168. 

%andreas
\bibitem{kleer} Kleer, J. de and Brown J.S. (1984)
deKleerBrown1984.pdf (4.005 Mb)
A qualitative Physics based on confluences, Artificial Intelligence, 24:7-83 (for the course you may ignore section 6)

%andreas
\bibitem{cioaca} Cioaca, Linnebank, Bredeweg, Salles 2009
Cioaca EtAlECOINF 2009.pdf (1,001.442 Kb)
A qualitative reasoning model of algal bloom in the Danube Delta Biosphere Resere (DDBR)
in: ecological informatics 4 (2009) p282-298

%hanne
\bibitem{buisman} Buisman, H., \emph{Automated modeling in process-based qualitative reasoning} \url{http://staff.science.uva.nl/~bredeweg/pdf/BSc/20062007/Buisman.pdf}

% (samenvatting van Buisman plus verbeteringen)
%hanne 
\bibitem{liem} Liem, J., Buisman, H. and Bredeweg, B., \emph{Supporting Conceptual Knowledge Capture Through Automatic Modelling}

%hanne
\bibitem{vanweelden} van Weelden, C., \emph{Automated modeling of conceptual knowledge} \url{http://staff.science.uva.nl/~bredeweg/pdf/BSc/20082009/vanWeelden.pdf}

\end{thebibliography}

\end{document}

----------------------------------------------

Notes

    * For the stacked bath tub model we changed the quantity-space of flow to ZPM (instead of ZP) because then we could replace the value-correspondences by a single Q-correspondence..
    * To be able to intergrate the splitting into the AM we need full scenario because we need all information about entities and quantities (we've entered them by hand for now..)

Ideas

    * done: split code for model output into generation & garp adding
    * current: compositional model fragments (also: accept existing model fragments as input)
    * deferred: lift full envisionment requirement, accept negative input (eg. list of impossible value combinations).
    * deferred: user interaction: graphical dialog to choose best model in case of ambiguity

----------------------------------------------

Minutes
April 13th 2010

New plan:

    * Some sort of standard for what a well-formed fragment should be
    * -- Static vs process fragments?
    * -- Everything per entity 1 fragment?
    * -- Things like 'birth', 'death', 'emigration' and 'immigration'?
    * -- Influences?

    * We need some sort of evaluation to check our models against.

    * First run a the models we want to use and analyze their current output, compare them to our 'ideal' output.
    * We should not worry about intergration with the current Automatic Modeling code for now. First focus on hand-corrected input.
    * Recognizing duplicate (super)cluster could be a nice step to start with.. to get working on the code.

Notes:

    * Influences: now written to screen --> should be asserted
    * I's wrong direction? (inf-pos-a-b = b --> a)
    * naive dependencies: exponential
    * Communicating vessels: hack in causal groups: same entities (same_object_group/2)
    * Choose hacks for conceptually best model

    * Discussed with Floris that finding correspondences could be done faster by first looking for them without considering naive_dependencies yet.


