\documentclass{article}
\usepackage{fullpage}
\usepackage{amsmath}
\usepackage{verbatim}
\usepackage[pdftex]{graphicx}
\usepackage[english]{babel}
\usepackage[colorlinks=true, linkcolor=black, urlcolor=blue, pdfborder={0 0 0}]{hyperref}
%\usepackage{synttree}
%\usepackage[utf8x]{inputenc}

\title{Model Induction in Qualitative Reasoning}
\author{Andreas van Cranenburgh \\ 0440949 \\ acranenb@science.uva.nl \and Hanne Nijhuis \\ 0364568 \\ hnijhuis@science.uva.nl}

\begin{document}
\maketitle

\begin{abstract}
\end{abstract}

\section{Introduction}

\section{Literature Review}
In this section we will review a few of the papers that have contributed greatly to the definition of Qualitative Reasoning. We will give a short summary of the papers and will conclude with an overview of the current field of research and possible future work.
\subsection{Qualitative Reasoning}
The following papers describe the fundaments of qualitative reasoning.
\subsubsection{A qualitative Physics based on confluences (de Kleer \& Brown)}
\subsubsection{Qualitative Process Theory (Forbus)}
\subsubsection{Garp3 - Workbench for Qualitative Modelling and Simulation (Bredeweg et al.)}

\subsection{Automated Modelling}
To introduce the recent developments in the field of automated modelling we will now review a few papers on this topic.
\subsubsection{Hylke}
Cluster, Superclusters, Causal paths.

\subsubsection{Jochem}


\subsubsection{Carsten}
Causal groups

\subsection{Overview}


\section{Theory}
In the field of QR, models ( / scenario's / systems ?) are build up out of several smaller fragments which can be roughly devided over 3 categories: static, process and agent. This categorization is not without debate, but gives a certain view on how things work. Static fragments are used to describe the structure of a model, as well as proportionalities between quantities. A static fragment cannot have any agents or influences in it. A \emph{process} fragment should have at least one direct influence. Finally, \emph{agent} fragments should be used to describe any influences on a system that are external, hence not part of the system itself.

Fragments can be useful to make the complete system more comprehensable. Also, while building these complex models, fragments allow the user to focus on what happens in a small part of the model, without having to worry about the other 'components'. For example in a model describing a population, seperate fragments could describe processes like birth, death, emigration, etc.

Recent work on automated modeling \cite{buisman, vanweelden, liem} has focused on generating a correct model based on a full-envisionment behavior graph. The current algorithms always produce a single modelfragment which describes the complete system. The useful fragmentation of a system is hereby lost and comprehending the output could become quite difficult with large models. We therefor introduce a way of splitting such a single monolithic model into smaller fragments while preserving the correct behavior.

\section{Approach}
In this section we describe our approach on inducing modelfragments.
\subsection{Monolithic model}
We take a monolithic model and divide it into several smaller fragments. To do so, we need to comply to several requirements:
\begin{itemize}
\item all dependencies have to be retained
\item the behavior has to be reproducable
\item 
\end{itemize}
\subsection{Algorithm}

\section{Experiments \& Results}
\subsection{Evaluated Models}
\subsubsection{Tree and shade growth}
\subsubsection{Stacked bathtubs}
\subsubsection{Communicating vessels}
\subsubsection{Population dynamics}

\section{Future work}

\section{Conclusion}

\begin{thebibliography}{99}

% 1,5 - 2 pp. per artikel

%allebei
\bibitem{bredeweg-eco} Bredeweg, B. and Salles, P. (2009), Handbook of Ecological Modelling, Chapter 19 - Mediating conceptual knowledge using qualitative reasoning. In: J\/orgen, S.V., Chon, T-S., Recknagel, F.A. (Eds.), Handbook of Ecological Modelling and Informatics. Wit Press, Southampton, UK, pp. 351.398.

%allebei
\bibitem{bredeweg-garp} Bredeweg, B., Linnebank, F., Bouwer, A. and Liem, J. (2009)
02 Bredeweg EtAl ECOINF 2009.pdf (598.325 Kb)
Garp3 - Workbench for Qualitative Modelling and Simulation. Ecological Informatics 4(5-6), 263-281.

%hanne
\bibitem{forbus} Forbus, K.D. (1984)
Forbus1984.pdf (4.108 Mb)
Qualitative process theory. Artificial Intelligence, 24:85-168. 

%andreas
\bibitem{kleer} Kleer, J. de and Brown J.S. (1984)
deKleerBrown1984.pdf (4.005 Mb)
A qualitative Physics based on confluences, Artificial Intelligence, 24:7-83 (for the course you may ignore section 6)

%andreas
\bibitem{cioaca} Cioaca, Linnebank, Bredeweg, Salles 2009
Cioaca EtAlECOINF 2009.pdf (1,001.442 Kb)
A qualitative reasoning model of algal bloom in the Danube Delta Biosphere Resere (DDBR)
in: ecological informatics 4 (2009) p282-298

%hanne
\bibitem{buisman} Buisman, H., \emph{Automated modeling in process-based qualitative reasoning} \url{http://staff.science.uva.nl/~bredeweg/pdf/BSc/20062007/Buisman.pdf}

% (samenvatting van Buisman plus verbeteringen)
%hanne 
\bibitem{liem} Liem, J., Buisman, H. and Bredeweg, B., \emph{Supporting Conceptual Knowledge Capture Through Automatic Modelling}

%hanne
\bibitem{vanweelden} van Weelden, C., \emph{Automated modeling of conceptual knowledge} \url{http://staff.science.uva.nl/~bredeweg/pdf/BSc/20082009/vanWeelden.pdf}

\end{thebibliography}

\end{document}
